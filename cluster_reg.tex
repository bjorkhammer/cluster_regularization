 \documentclass[%
 aps,
 prl,%
 amsmath,amssymb,
%preprint,%
 reprint,%
%author-year,%
%author-numerical,%
]{revtex4-1}

\usepackage{graphicx}% Include figure files
\usepackage{dcolumn}% Align table columns on decimal point
\usepackage{bm}% bold math
%My packages start
%\usepackage{subfig}
\usepackage{multirow}					
\usepackage{array}
%\usepackage{booktabs}
\usepackage{footnote} 
\newcommand{\angstrom}{\text{\normalfont\AA}}
\usepackage{booktabs}
\usepackage{float}
\usepackage{mathtools}
\usepackage{color}
\newcommand{\comment}[1]{\noindent \textcolor{blue}{#1}}
%My packages end
%\usepackage[mathlines]{lineno}% Enable numbering of text and display math
%\linenumbers\relax % Commence numbering lines

\begin{document}

\title[]{Accelerating atomic structure search with cluster regularization}% Force line breaks with \\

\author{K.\ H.\ S{\o}rensen}
\author{M.\ S.\ J{\o}rgensen}
\author{B.\ Hammer}
\email{hammer@phys.au.dk}
\affiliation{ 
Department of Physics and Astronomy, and Interdisciplinary Nanoscience Center (iNANO), Aarhus University, DK-8000 Aarhus C, Denmark.
}
% \homepage{phys.au.dk}
 
\date{\today}% It is always \today, today,
             %  but any date may be explicitly specified

\begin{abstract}
We present a method for accelerating the global structure optimization
of atomic compounds. The method is demonstrated to speed up the
finding of the anatase TiO$_2$(001)-(1$\times$4) surface
reconstruction within a tight-binding density functional theory (DFTB)
based evolutionary algorithm (EA) framework. As a key element of the
method, we use unsupervised machine learning techniques to categorize
the atoms in a diverse set of partially disordered surface structures
into clusters of atoms having similar local atomic
environments. Analysis of more than 1000 different structures show
that the total energy of the structures correlate with the summed
distances in feature space of the atomic environments to their
respective cluster centers, where the sum runs over all atoms in each
structure. Our method is formulated as a gradient based minimization
of this summed cluster distance for a given structure and alternates
with a standard gradient based energy minimization. While the latter
minimization ensures local relaxation within a given energy funnel,
the former enables escapes from meta-stable funnels and hence increases
the overall performance of the global optimization.
\end{abstract}

\pacs{Valid PACS appear here}% PACS, the Physics and Astronomy
                             % Classification Scheme.
\keywords{}%Use showkeys class option if keyword
                              %display desired
\maketitle

\section{\label{sec:introduction}Introduction}
In computational materials science, knowing the atomic structure of a
given molecule, atomic cluster or solid compound is a prerequisite for
further prediction of electronic and thermodynamic properties of such
a substance. In the emerging fields of combinatorial chemistry and
high-throughput computational screening of materials, the use of local
relaxation of probed structures is often sufficient since libraries of
molecular building blocks and crystal structures can be used to direct
the starting points for the searches. However, many problems exist for
which the structural motifs in the sought-after structures have no
analogues in known structures, and where global optimization must be
employed. A prominent example of this is that of surface relaxations,
that often exhibit structural motifs that are unique to a chemical
composition, crystaline polymorph, and surface orientation. Monomeric
Si adatoms and restatoms and zig-zagging Si-dimers do for example evolve at the
Si(111)-(7$\times$7) and Si(001)-c(4$\times$2) surfaces, respectively, but
are otherwise not present in bulk Si or bulk-truncated Si surfaces. Likewise,
rutile and anatase TiO$_2$ single-crystal surfaces are known to exhibit
reconstructions and ad-structures, that have no equivalent in any TiO$_2$ bulk or surface systems.

Many strategies exist for performing global optimization in
conjunction with model potential or first-principles total energy
frameworks. Among these, stoichastic methods such as random search and
basin hopping are widely used and more advanced methods based on
evolutionary algorithms (EA) are becoming increasingly popular. Common
to these methods is the need for the perturbative update steps
followed by local relaxation and evaluation of whether or not to
retain the new structure. The nature of the perturbation steps ranges
from a mere rattling of the atomic positions to highly advanced
cross-over operations in which the atomic structures of several known
systems are combined. The exploration of configuration space is
ensured by the very stochastic nature of these updates, yet updates
that are too random will lead to rejection of the locally
relaxed structure too often and hence cause slow convergence. The atomic
displacement amplitude in a rattling update is a good example of
this. The amplitude must be kept small and sometimes only apply to a
subset of the atoms, as the new structural candidate otherwise become
too unstable.

The present work aims at formulating a means of performing a random
update in a way that optimizes the chances of finding more stable
structural candidates in subsequent local relaxation steps. The method
proposed starts from assigning every atom in a given structure to a
cluster of similar atoms present in a reference set of structures.
The similarity is measured as the distance in a feature space, chosen in this work
sufficiently simple that it can be illustrated. With every atom
assigned to a cluster, we evaluate for a given structure the sum of
the distances in feature space of the atoms from their respective
cluster centers. This structure specific scalar measure of distance to
cluster centers is demonstrated to correlate with the structure
stability -- the lower the distance measure, the more stable the
structure. As a consequence, minimizing the cluster distance measure
contains an element of bringing a structure into a region of
configuration space that is expectedly more stable. The minimization
can be done by moving opposite to an analytic gradient and has
potential to take a structure out of local energy minima, since the
cluster distances are measured in a space that is complementary to that of the
energy landscape. We refer to our method as the
\textit{cluster regularization method} as it penalizes large cluster
distances.

The paper is outlined as follows: In the Methods section we present
the computational setup, outline the reference method used, and
describe the required machine learning components of our method,
including the choice of feature vector and the clustering technique.
In the New Methods section we analyse some data and formulate the cluster
regularization method. In the Results and Discussion
section, the method is used in full scale global optimization in
search for the anatase TiO$_2$(001)-(1$\times$4) surface
reconstruction. This section demonstrates the usefulness of the method
and further investigates its efficacy as the number of unknown atomic
position is increased in the global optimization. The paper ends with
a Conclusion section.



\section{Method}

\subsection{Tight-binding density functional theory}
Tight-binding density functional theory calculations (DFTB) were
performed for TiO$_2$ structures using parameters from Ref. \cite{}.
With this description, bulk anatase TiO$_2$ has lattice parameters of
XX and XX {\AA} that compare well with the experimental values of XX
and XX {\AA}. DFTB calculations have previously been conducted
successfully for the study of XX and XX.  In the present work, we
employ 2D periodic super cells accomodating the anatase TiO$_2$(001)
surface with (4$\times$1) periodicity. Slabs of three different
thicknesses were considered, containing two, three, or four layers of
atoms. The three-layer system is illustrated in Fig.\ \ref{figintro},
showing the template of four static TiO$_2$ units, two examples of
disordered structures with nine extra TiO$_2$ units added, and the
global minimum energy structure with three well ordered atomic layers
and one extra row of TiO$_2$ protruding out of the surface.
\begin{figure}[tb]
    \centering
    \includegraphics[width=1.0\columnwidth]{fig1-intro.pdf}
    \caption{Side views of computational cells with Ti$_{n}$O$_{2n}$ structures.
      Small red spheres: Oxygen, large grey spheres: Ti.
      a) The template of four TiO$_2$ units.
      b-c) Two examples of intermediate structures from an evolutionary search. Nine TiO$_2$ units
      have been placed on the template.
      d) The sought-after global minimum structure for the anatase TiO$_2$(001)-(1$\times$4) surface.
    }
    \label{figintro}
\end{figure}

\subsection{Evolutionary algorithm}
The disordered structures in Fig.\ref{figintro} are snapshots from
search runs performed with the evolutionary algorithm (EA) available
in the Atomic Simulation Environment (ASE)\cite{ase2}.  The EA
iteratively improves the positions of the atoms atop the template
structure (Fig.\ \ref{figintro}a) using cross-over operations
combining two parent structures or mutation operations applied to
single parent structures. The resulting offspring structures undergo
local relaxation within DFTB following the atomic forces (i.e.\ minus
the energy gradient). A population of $N_{pop}=20$ parent structures
is maintained while the up to $N_{run}=20$ offspring structures are
evolved asyncronously on a parallel high performance super
computer. The relaxed offspring structures are included in the population
according to their fitness evaluated as minus the DFTB energy.
More details on the EA are given in Refs.\ \cite{Vilhelmsen2012,Vilhelmsen2014}.

The cluster regularization method proposed in this work is implemented as
a mutation operation and its usefulness is gauged by comparing an EA search
where this mutation is used either exclusively or occationally to a benchmark
EA search, where it is not used at all.

\subsection{Machine learning techniques}
Our method uses two components of general machine learning techniques:
the representation of the data with a feature vector, and the
classification of the data utilizing clustering methods.

\subsubsection{Feature vector}
Feature vector representations of the atomic structure of molecules,
clusters, and solids serve the general purpose of introducing
invariance with respect to symmetries obeyed by the total energy
operator, i.e.\ Hamiltonian. The symmetries are those of translation
or rotation of the entire system or the permutation of the order of
atoms with the same chemical identity. Without a symmetry invariant
representation, e.g.\ using cartesian coordinates, a certain structure
will have a finite distance to for instance a translated copy of
itself, even though the two systems will have identical physical
properties. However, when measured in a proper feature space, two such
structures may have a zero distance.

A large and increasing number of feature vector formulations appear in
the chemical physics literature. Global feature vectors of entire
systems include the bag-of-bonds\cite{} and fingerprint\cite{} feature
vectors, that are composed of either a sorted list or a histogram of
all interatomic distances in the compounds, but also more
sophisticated methods such as the Coulomb matrix\cite{} have been
proposed, though lacking the important permutation
invariance.

Atom-specific feature vectors representing the local
chemical environments of the atoms in a compound structure are
particularly abundant in the literature. Here the Behler-Parrinello
symmetry function formulation\cite{} stand out as a seminal
contribution, yet one with rather many parameters to be chosen by the
user. With symmetry functions, each atom is represented by a selection
of 2-body distance terms and 3-body angular terms. All terms are
attenuated with an envelope function that ensures a smooth decay to
zero at a set cutoff distance.  Recently, more advanced atom-specific feature vectors
have been developed in which e.g.\ 4-atom torsion angles are included.
Such feature vectors have proven superior in reproducing energies,
HOMO-LUMO gaps, polarizations, and a number of other properties for a
diverse set of molecules.\cite{}

In the present work we adopt an exceedingly simple atom-specific
feature vector with only three components:
\begin{equation}
\vec{f}_i=\left[Z_i,\rho_i^{Z^\mathrm{O}},\rho_i^{Z^\mathrm{Ti}}\right]\label{eq_feature}
\end{equation}
where $Z_i$ is the atomic number of atom $i$, while
$\rho_i^{Z^\mathrm{O}}$ and $\rho_i^{Z^\mathrm{Ti}}$ are measures for
atom $i$ of the density of neighboring oxygen and titanium atoms,
respectively. These densities are inspired by the radial symmetry
functions of Behler \textit{et al.},\cite{} for a dicussion see the work of Botu
and Ramprasad \cite{Botu2015}. In our work we use:
\begin{equation}
\rho_i^Z = \sum_{j\ne i,Z_j=Z} e^{-r_{ij}/\lambda}g_c(r_{ij})  \label{eq1}
\end{equation}
where $Z$ is either $Z^\mathrm{O}$ or $Z^\mathrm{Ti}$, $j$ runs over
all atomic indices, $Z_j$ is the atomic number of the $j$'th atom, $r_{ij}$ is the distance $|\vec r_{ij}|=|\vec r_j-
\vec r_i|$ from the $i$'th to the $j$'th atom, and
$\lambda=1\text{\AA}$ is a chosen length scale. $g_c$ is a cut-off
function given by:
\begin{equation}
g_c(r)=
\left\{\begin{array}{ll}
\frac{1}{2}{\cos(\pi \frac{r}{r_c})+1},&r<r_c\\
\\
0,&r\ge r_c
\end{array}
\right.
\end{equation}
which ensures that the densities vanish algebraically at the cut-off
radius, $r_c$. Owing to the presence of $g_c$, the sum in Eq.\ (\ref{eq1}) needs
in practice only to run over nearest neighbors to atom $i$.

\subsubsection{Clustering}
XXX Some general comments about clustering. Reference to use of clustering in Chemical Physics.

The clustering method used is this study is k-means with two
modifications. The first is a modification to avoid generating empty
clusters \cite{Malay2009} and the second modification is the use of
k-means++ cluster initialization. Unless otherwise stated, we use a fixed number
of cluster centers, $N_c=5$.

Whenever a clustering of atom-specific feature vectors has been
performed, a set of resulting cluster centers,
$\left\{\vec{c}_k\right\}$, will be known. Representing by $k(i)$ the
index of the cluster center that a given atom $i$ belongs to, we can
now evaluate the sum of all cluster distances for a given structure accoring to:
\begin{equation}
D_S = \sum_{i \in S} \left|\vec f_i- \vec c_{k(i)}\right| \label{eq3}
\end{equation}
where $S$ is the structure (a list of atoms). Note that $D_S$ is a
scalar number, which in principle exists for any conceivable set of
$N$ atomic positions. As such, a continuous
"cluster-distance"-landscape exists in $\mathbb{R}^{3N}$, and the
atomic positions may be changed towards an overall smaller cluster
distance by following minus the gradient of the cluster distance with
respec to the atomic positions, $\mathbf{R}$:
\begin{equation}
-\nabla_\mathbf{R} D_S = -\sum_{i \in S} \nabla_\mathbf{R}\vec{f}_i
\quad XXXXXXXX\quad
\nabla_{\vec{f}_i}\left|\vec f_i- \vec c_{k(i)}\right|,
\end{equation}
where the cluster centers are kept fixed.  This is analogous to the
atomic forces directing the minimization of the energy in the energy
landscape.



\section{New method}

\subsection{Analysis}
Owing to its low dimensionality, the atom-specific feature vector can
be illustrated in two-dimensional scatter plots. This is done in
Figs.\ \ref{fig_global} and \ref{fig_feature} where features
pertaining to oxygen and titanium atoms are shown in red and grey
color, respectively.  The two axes in the plots are the
$\rho^{Z^\mathrm{O}}$ and $\rho^{Z^\mathrm{Ti}}$ densities.  In
Fig.\ \ref{fig_global}, the atomic structure considered is that of the
three-layer global minimum energy structure introduced in
Fig.\ \ref{figintro}d. The figure shows how strong correlations exist
for the features. Features for oxygen atoms lie on one line, while
features for titanium atoms lie on another. Within each line the 26
data points for oxygen atoms and 13 data points for titanium atoms
further appear in a number of clusters. The lines reflect that in this
global minimum energy structure the atoms have a nearly constant
ratio of neighboring atoms of the two possible types.

\begin{figure}[tb]
    \centering
    \includegraphics[width=1.0\columnwidth]{fig2-gm_scatterplot.pdf}
    \caption{Visualization of global minimum for Ti$_{13}$O$_{26}$ structures in feature space}   
    \label{fig_global}
\end{figure}



\begin{figure}[tb]
    \centering
    \includegraphics[width=1.0\columnwidth]{fig3-scatterplot.pdf}
    \caption{Visualization of atoms from 1019 Ti$_{13}$O$_{26}$ structures in feature space}   
    \label{fig_feature}
\end{figure}

Figure \ref{fig_feature} presents the distribution of atomic features
for the 26+13 atoms in about 1000 different structures of the
three-layer system. The structures are taken from several EA searches
and have been filtered so that they can be considered distinct
structures. The wide distribution of the feature vectors for general structures as opposed to
the highly ordered 

\begin{figure}[tb]
    \centering
    \includegraphics[width=1.0\columnwidth]{fig4-correlation.pdf}
    \caption{Cluster distance vs. energy correlation}
    \label{fig_corr}
\end{figure}

Introducing clustering of the atomic feature vectors present in
Fig.\ \ref{fig_feature} a strong correlation between, the total
cluster distance, $D_S$, and the total DFTB energy,
$E_S^\mathrm{DFTB}$, is revealed. The correlation is illustrated in
Fig.\ \ref{fig_corr}. The correlation may in part be explained by the
general observation that Nature often favors high
symmetry\cite{Pikard2011}, i.e.\ structures with low energy often has
recurring atomic motifs. A single crystal is an obvious example of
this, in the ground state, the atoms all assume the same local
environments. In fact, the scatter plot similar to
Figs.\ \ref{fig_global} and \ref{fig_feature} only contains one point
for each of the two chemical elements, oxygen and titanium, and with
two cluster centers, the total cluster distance would be zero. In our
case, a slab geometry is used causing the presence of two surfaces
that act to give several possible motifs, as evidenced for the global
minimum by Fig.\ \ref{fig_global}, where now the total cluster
distance becomes finite, reflecting the surface energy cost.  For any
disordered slab structure, the feature vectors scatter even more and
(according to Fig.\ \ref{fig_corr}) the cluster distance increases
further at the same time as the total energy of the structure
increases above that of the global minimum energy structure.

\section{Cluster regularization method}
The observed correlation between energy and cluster distance forms the
basis for our new method, \textit{Cluster regularization}. The method


Then we switch and minimize the energy with DFTB+ and then minimize cluster distance again. Figure \ref{fig_min} shows the process. The idea is that alternate the two minimizations they will help each other out of there local minima and further down in energy and cluster distance.
   
\begin{figure}[h]
    \centering
    \includegraphics[width=1.0\columnwidth]{fig5-minimize.pdf}
    \caption{Green lines and arrows shows a cluster distance (cd) minimization and the blue DFTB+ relaxations. Here the first/bottom layer is the template. While it is hard to see progress in individual minimization step there is small details to notice on the plot. Notice that in the cd. step from 16 to 17 the hole in the second layer is filled and that in the cd. step from 18 to 19 the 2 oxygen atoms in top left corner is split up.}
    \label{fig_min}
\end{figure}


\section{Results and Discussion}

\subsection{Test setup}
To test the technique we implemented cluster regularization as a mutation operator in ASE's genetic algorithm (ga)\cite{ase_ga} framework. 
The mutation operator takes a list of parents and the execute the following steps:
\begin{enumerate}
\item Calculate atomic features for all parents.
\item Cluster them and find cluster centers.
\item Copy the lowest energy parent to the child
\item minimize cluster distance of child.
\item return child.
\end{enumerate}
When the mutation returns the child then the ga will make a energy relaxation 
and we have a cycle as illustrated in figure \ref{fig_min}.

To establish a benchmark the ga with bayesian hyper-parameter optimization (BHO.) search was run on the 3 layer system.
Run with cut\&splice, permutation and rattle mutations the search found the optimal ratios was 59\% cut\&splice, 0\% permutation, 41\% rattle. 

Next we did a BHO. search for the best parameters for cluster regularization, we found that 5 clusters, 3 parents and cutoff radius c=11.9 gave the best results.

Last we did a BHO. search for the best combination of cluster regularization, cut\&splice and rattle, we found that 70\% cluster regularization, 28\% cut\&splice and 2\% rattle did well.

We tested these 3 methods on a 2 layer, 3 layer and a 4 layers system. 

\subsection{Test result}

On figure \ref{fig_success} we see that pure cluster reg. (orange curve) start finding minimum structure much faster than the benchmark (yellow curve) , but it levels out around 60-70\%. The problem is that cluster regularization and energy minimization have local minima close to each other and running the algorithm make them switch between this minima. When cluster regularization is mix with other mutations (red curve) the problem disappear. 

\begin{figure}[h]
    \centering
    \includegraphics[width=1.0\columnwidth]{fig6-success.pdf}
    \caption{Cumulative success for 2,3 and 4 layer systems}
    \label{fig_success}
\end{figure}

With the 3 layers system on figure \ref{fig_success} the problem isn't as prominent which might be because there is more free variables in this problems. Else we observe the same as the 2 layer system combined and pure cluster reg. does better than the benchmark. Pure cluster reg. does better in the start but is overtaken by the combined method later on. 
In the 4 layer system the tendency continues but the success rate is much lower and we made double as many runs to shrink the confidence intervals. 













\section{Conclusion}

We acknowledge support from the Danish Council for Independent Research | Natural Science (grant no. 0602-02566B) and from VILLUM FONDEN (Investigator grant, project no. 16562).

\begin{thebibliography}{12}  
\bibitem{Starke1998} U. Starke \textit{et al}., Phys. Rev. Lett. \textbf{80}, 758 (1998).    
\bibitem{Malay2009} {A Modified k-means Algorithm to Avoid Empty Clusters} Malay K. Pakhira, International Journal of Recent Trends in Engineering, Vol 1, No 1, May 2009
\bibitem{Pikard2011}  {Ab initio random structure searching} Chris J. Pickard and R. J. Needs. J. Phys.: Condens. Matter \textbf{23} (2011) 053201 (23pp) DOI:10.1088/0953-8984/23/5/053201
\bibitem{Botu2015} {Adaptive Machine Learning Framework to Accelerate Ab Initio Molecular Dynamics.} V. Botu and R. Ramprasad, Int. J. Quantum Chem. 2015, 115 ,1075-18083. DOI:10.1002/qua.24836    
\bibitem{Chen2017} {Understanding atomic chemistry with machine learning.} Chen, X., J�rgenson, M. S., Hammer, B., \& Li, J. (2017) Unpublished
\bibitem{Vilhelmsen2012}L. B. Vilhelmsen and B. Hammer, Phys. Rev. Lett. \textbf{108}, 126101 (2012).
\bibitem{Vilhelmsen2014}L. B.\ Vilhelmsen and B.\ Hammer, J. Chem. Phys. \textbf{141}, 044711 (2014).
\bibitem{ase2}A.\ H.\ Larsen \textit{et al.}, J. Phys. Condens. Matter 2017, 29, 273002.
\end{thebibliography}

\section{removed sections}

It is a way to impose symmetry into the search without specifying the symmetry explicit, 
as pointed out in \cite{Pikard2011} low energy structures tend to have a very high degree of symmetry.

The method first constructing a feature vector for each atom, then cluster this feature vectors, and at last update the atomic positions such that the distance between the feature vectors and the cluster centers are minimized.   



\end{document}


